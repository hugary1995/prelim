\subsection{Incorporating thermal effects}

\subsectioncover

\begin{frame}{}
\contenttitle{Neumann-type boundary conditions} can be defined based on an approximated crack surface normal:
\begin{align}
    \int\limits_\bodyboundary \bs{t} \dArea \approx \int\limits_{\body_t} \bs{t} \norm{\grad_{\bs{x}} d} \dVolume = \int\limits_{\body_0} \bs{T} \norm{\grad d} \dVolume,
\end{align}
where $\bs{t}$ is some surface flux and $\bs{T} = \defgrad^{-1} \cdot \bs{t}$ is its material form. \\
\bigskip
$\Gamma$-convergence of this approximation has been sketched in \cite{chukwudozie2019variational}. \\
\bigskip
A numerical investigation of the $\Gamma$-convergence can be found in \cite{nguyen2017phase} in the context of finite cell method. \\
\bigskip
\cite{miehe2015phase} provides a similar approximation based on finite-differencing arguments. \\
\bigskip
Many studies on pressurized crack and hydraulic fracture have used a special case of the approximation where $\bs{t} = p_0\xnormal$ with $p_0$ being a scalar-valued pressure field.
\end{frame}

\begin{frame}{}
\contenttitle{A better approximation of the crack surface normal} \\
The closed-form solution to the phase-field in 1D, depending on the specific choice of the local fracture dissipation function, is
\begin{align}
    d(\tau; l) =
    \begin{cases}
        \left( 1 - \dfrac{\tau}{2l} \right)^2, \quad w(d) = w_l(d), \\
        1-\exp(-\tau/l), \quad w(d) = w_q(d),
    \end{cases}
\end{align} \\
\bigskip
A Cahn-Hilliard approximation to the crack surface density takes the form
\begin{align}
    \gamma_4(d; l) = \dfrac{1}{c_0 l} \left[ w(d) + 2 l^2 \grad d \cdot \grad d + l^4 (\Delta d)^2 \right],
\end{align}
and a $C^\infty$ closed-form solution exists when $w(d) = w_q(d)$, i.e.
\begin{align}
    d(\tau; l) = \exp(-\tau/l) \left( 1 + \dfrac{\tau}{l} \right),
\end{align}
and the approximation of the crack surface normal will be smooth almost everywhere.
\end{frame}

\begin{frame}{}
\contenttitle{Thermal effects} \\
Disregarding the adiabatic assumption and taking into account the aforementioned approximation to heat flux, the modified heat equation can be written as
\begin{align}
    g(d) \rho_0 c \dot{T} = g(d) \rho_0 r - g(d) \divergence \kappa \grad T - \dfrac{\partial \pi}{\partial \plasticstrain}\dot{\plasticstrain} - \dfrac{\partial \pi}{\partial d} \dot{d},
\end{align}\\
\bigskip
Thermal-elastic coupling can be introduced by an inelastic potential $r(T;T_0,\alpha) \tr(\strain)$ where $T_0$ is the reference temperature and $\alpha$ is the thermal expansion coefficient. \\
\bigskip
Heat generation due to plasticity has been considered in the heat equation. Thermal softening in the plastic regime can be defined by the degradation of hardening modulus as a function of temperature $h = \widetilde{h}(T)$.
\end{frame}
