\subsection{Proposal}

\begin{frame}{}
\contenttitle{Challenge:}
To predict material behavior with the presence of pervasive fracture. Cracks may nucleate, propagate, branch, and coalesce. \\
\bigskip
\pause
\contenttitle{Techniques:}
Element deletion, cohesive element insertion, mesh fragmentation, enrichment methods (XFEM, GFEM, etc.), damage models (local damage, gradient damage, phase-field), and counting...\\
\bigskip
\pause
\contenttitle{Current state of phase-field methods} \\
Elastic-fracture coupling: \cite{Francfort98, Bourdin2000, bourdin2008variational, karma_2001, karma_2004, henry_2004, spatschek_2007, amor_2009}. \\
Hyperelastic-fracture coupling concerning tension-compression asymmetry: \cite{borden2012isogeometric, zhang2016variational}. \\
Tight Thermal-fracture coupling: \cite{miehe2015phase, ulmer2013phase}. \\
Plastic-fracture coupling: \cite{alessi2018coupling, ambati2016phase, borden2016phase, duda2015phase, kuhn2016phase, miehe2016phase}. \\
Hyperelastic-plastic-thermal-fracture coupling: \cite{miehe2015phase} \\
\bigskip
\pause
\contenttitle{Is it good enough?}
Sandia Fracture Challenge since 2012. Variational fracture models are not always the winner. Why... \\
\bigskip
\pause
\contenttitle{Objective:}
I propose to extend the state-of-the-art phase-field models to simulate large deformation ductile fracture with thermal effects under dynamic loading.
\end{frame}
