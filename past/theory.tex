\subsection{Theory}

\subsectioncover

\begin{frame}{}
\begin{figure}[htb!]
    \centering
    \begin{overpic}[scale=0.35]{past/figures/motion.png}
        \put(41,21){$\bs{\phi}(\bs{X}, t)$}
        \put(0,30){$\body$}
    \end{overpic}
\end{figure}

\begin{columns}
    \begin{column}{0.33\textwidth}
        \contenttitle{Kinematics} \\
        $\bs{X}$: material point  \\
        $\bs{x}$: spatial point \\
        $\body$: reference configuration \\
        $\bodyboundary$: external boundary \\
        $\crackset$: permanent crack set \\
        $\bs{\phi}(\bs{X}, t)$: deformation map \\
        $\defgrad = \partial \bs{x}/\partial \bs{X}$: deformation gradient \\
        $J = \det(\defgrad)$ jacobian
    \end{column}
    \begin{column}{0.33\textwidth}
        \contenttitle{Observable quantities} \\
        $T$: temperature \\
        $\defgrad$ deformation gradient \\
        \bigskip
        \contenttitle{Internal variables} \\
        $\defgrad^p$: plastic deformation gradient \\
        $\plasticstrain$: effective uniaxial plastic strain \\
        $d$, $\grad d$: phase-field
    \end{column}
    \begin{column}{0.33\textwidth}
        \contenttitle{State} \\
        $\mathcal{S} = \{ T, \defgrad \}$ \\
        $\mathcal{X} = \{ \defgrad^p, \plasticstrain, d, \grad d \}$ \\
        $\pi = \rho_0 \psi$: Helmholtz free energy
    \end{column}
\end{columns}
\end{frame}

\begin{frame}{}
\contenttitle{Balance laws} \\
Mass balance (no mass transport):
\begin{align}
    J\rho = \rho_0
\end{align} \\
Linear momentum balance:
\begin{align}
    \divergence \bs{P} + \rho_0 \bs{b} = \bs{0}
\end{align} \\
Angular momentum balance:
\begin{align}
    \bs{P} \defgrad^T = \defgrad \bs{P}^T
\end{align} \\
Eenergy balance (1st law, adiabatic):
\begin{align}
    \bs{P} : \dot{\defgrad} + \rho_0 r = \dot{\pi} + \rho_0 T \dot{s} + \rho_0 \dot{T} s \\
    \rho_0 c \dot{T} = \rho_0 r - \pi_{, \plasticstrain}\dot{\plasticstrain} - \pi_{, d} \dot{d}
\end{align} \\
Entropy production (2nd law, or Clausius-Duhem inequality):
\begin{align}
    -\dot{\pi} - \rho_0 s \dot{T} + \bs{P} : \dot{\defgrad} \geqslant 0
\end{align}
\end{frame}

\begin{frame}{}
\contenttitle{Constitutive constraints} \\
The specific entropy:
\begin{align}
    s = -\psi_{, T}
\end{align} \\
The first Piola-Kirchhoff stress:
\begin{align}
    \bs{P} = \pi_{, \defgrad}
\end{align} \\
The dissipation inequality:
\begin{align}
    - \pi_{, \defgrad^p} : \dot{\defgrad^p} - \pi_{, \plasticstrain}\dot{\plasticstrain} - \pi_{, d} \dot{d} - \pi_{, \grad d} \cdot \grad \dot{d} \geqslant 0,
\end{align}
which, by the principal of maximum dissipation, gives me the equations of flow rule, yield surface and fracture evolution.
\end{frame}

\begin{frame}{}
In the soil desiccation problem, the free energy is written as
\begin{align}
    \pi = \pi_\elastic + \pi_\inelastic + \pi_\interface + \pi_\fracture.
\end{align}

In the UO$_2$ fracture problem, the free energy is written as
\begin{align}
    \pi =
    \begin{cases}
        \pi_\elastic^\text{bnd} + \pi_\fracture^\text{bnd}, \quad \forall \bs{X} \in \body^\text{bnd}, \\
        \pi_\elastic^\text{grain} + \pi_\fracture^\text{grain}, \quad \forall \bs{X} \in \body^\text{grain}.
    \end{cases}
\end{align}

In the ductile fracture model, the free energy takes the following form:
\begin{align}
    \pi = \pi_\elastic + \pi_\plastic + \pi_\fracture.
\end{align}

The free energy $\pi$ is polyconvex and coercive. We adopt an alternating minimization technique to address stability issues (see e.g. \cite{hong2017linear} for a convergence proof).
\end{frame}

\begin{frame}{}
Variational phase-field models (as a subclass of the gradient-damage models) rely on a relatively unified description of the fracture energy:
\begin{align}
    \pi_\fracture &= \Gc \gamma(d; l), \\
    \gamma(d; l) &= \dfrac{1}{c_0 l} \left[ w(d) + 2l^2 \grad d \cdot \grad d \right],
\end{align}
where $c_0$ is a normalization constant such that $\lim_{l \searrow 0} \gamma(d; l) = \delta(\crackset)$. \\
\bigskip
\begin{columns}
    \begin{column}{0.5\textwidth}
        \begin{block}{\textbf{Cohesive fracture}}
        \vspace{-1em}
        \begin{align*}
            w_1(d) &= d, \\
            g_{ql}(d) &= \dfrac{1 - d}{(1 - d) + m d}, \quad m = \dfrac{\Gc}{\pi_c} \dfrac{1}{c_0 l}, \\
            g_{qq}(d; p) &= \dfrac{(1 - d)^2}{(1 - d)^2 + m d (1 + p d)}, \quad m = \dfrac{\Gc}{\pi_c} \dfrac{1}{c_0 l} \geqslant p + 2.
        \end{align*}
        \end{block}
    \end{column}
    \begin{column}{0.5\textwidth}
        \begin{block}{\textbf{Brittle fracture}}
        \vspace{-1em}
        \begin{align*}
            w_2(d) &= d^2, \\
            g_q(d) &= (1 - d)^2.
        \end{align*}
        \end{block}
    \end{column}
\end{columns}
\end{frame}
